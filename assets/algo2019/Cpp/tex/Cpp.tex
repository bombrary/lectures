\documentclass[dvipdfmx,12pt]{beamer}% dvipdfmxしたい
\usepackage{bxdpx-beamer}% dvipdfmxなので必要
\usepackage{pxjahyper}% 日本語で'しおり'したい
\usepackage{minijs}% min10ヤダ
\usefonttheme{professionalfonts}
\renewcommand{\kanjifamilydefault}{\gtdefault}% 既定をゴシック体に
\usepackage{listings}
\lstset{
  language={C++}, %プログラミング言語によって変える。
  basicstyle={\ttfamily\tiny},
  keywordstyle={\color{blue}},
  commentstyle={\color{green}},
  stringstyle=\color{red},
  tabsize=2,
  %% breaklines=true, %折り返し
}

% あとは欧文の場合と同じ
\usetheme{default}
\title{もっとC++}
\author{山本 陸}


\begin{document}
\frame{\maketitle}

\begin{frame}[fragile]{ブロック環境を意味のまとまりとして使う}
  \begin{lstlisting}
void showVec(vector<int> *v) {
  for (int i = 0; i < v.size(); i++) {
    cout << v[i] << endl;
  }
}
int main()
{
  vector<int> v = {1, 2, 3, 4};
  showVec(&v);
  return 0;
}
Point operator+(const Point &a, const Point &b) {
  return Point(a.x + b.x, a.y + b.y);
}
Point operator-(const Point &a, const Point &b) {
  return Point(a.x - b.x, a.y - b.y);
}

int main()
{
  Point p1(1, 2), p2(3, 4);
  Point p3(p1 + p2);
  cout << p3.x << ' ' << p3.x << endl;
  return 0;
}
  \end{lstlisting}
\end{frame}

\frame{\centering \Large Thank you for listening}
\end{document}
